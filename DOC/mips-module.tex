\documentclass[11pt,a4paper]{article}

%\addtolength{\voffset}{-6mm}
%\addtolength{\textheight}{12mm}
%
%\addtolength{\evensidemargin}{-10mm}
%\addtolength{\oddsidemargin}{-2mm}
%\addtolength{\textwidth}{10mm}

\usepackage{graphicx}

\usepackage{amsmath}
\usepackage{amssymb}
\usepackage{amsthm}
\usepackage{array}

\usepackage[utf8]{inputenc}
\usepackage[T1]{fontenc}
\usepackage{lmodern}

\usepackage{fancyvrb}
\usepackage[normalem]{ulem}

\usepackage[linkcolor=black,colorlinks=true]{hyperref}

%%%%%%%%%%%%%%%%%%%%%%%%%%%%%%%%%%
%%%%%  reviewer comments  %%%%%%%%
%%%%%%%%%%%%%%%%%%%%%%%%%%%%%%%%%%
\newcommand{\comm}[2]{{\sf \(\spadesuit\){\bf #1: }{\rm \sf #2}\(\spadesuit\)}}
%\renewcommand{\comm}[2]{}
%%%%%%%%%%%%%%%%%%%%%%%%%%%%%%%%%%
\newcommand{\jbcomment}[1]{\comm{JB}{#1}}

%%%%%%%%%%%%%%%%%%%%%%%%%%%%%%%%%
%%%%%    code sections   %%%%%%%%
%%%%%%%%%%%%%%%%%%%%%%%%%%%%%%%%%

\newcommand{\codesize}{\scriptsize}

% code citation in text
\newcommand{\cd}[1]{{{\codesize\tt #1}}}
\newcommand{\xcd}[1]{{{\small \tt #1}}}
\renewcommand{\FancyVerbFormatLine}[1]{#1}

% code highlighting commands in own block
% code which is fed into GHC when file is loaded as *.lhs:
\DefineVerbatimEnvironment{code}{Verbatim}{fontsize=\codesize}

% Fancy code with color commands:
\DefineVerbatimEnvironment{colorcode}%
        {Verbatim}{fontsize=\codesize,commandchars=\\\{\}}
% use like this:
%% \begin{colorcode}
%% foo arg arg' = \emph{bar} {\large \{arg' \}} \textit{arg}
%% \end{colorcode}


\title{MIPS, Register allocation, and the MARS simulator}
%\title{MIPS, registerallokering og MARS}

\author{Jost Berthold\\ (based on Danish version by Torben Mogensen, 2010)}
%\author{Torben Mogensen}

\date{November 2012, updated November 2013}

\begin{document}
\setlength{\parindent}{0.2em}

\maketitle


\section*{Overview}
\addcontentsline{toc}{section}{Overview}

This document describes back-end tools and given SML modules for assignments 
in the lecture "Compilers" in the 2012 run: the MIPS instruction (sub-)set,
a register allocation module, and the simulator MARS to execute MIPS assembly code.

\tableofcontents

\section{MIPS Instruction Set Module}

The module \cd{Mips.sml} defines a data structure to describe a subset of the 
MIPS\footnote{MIPS stands for \emph{Microprocessor without 
Interlocked Pipeline Stages}, a microprocessor in RISC (reduced instruction set)
architecture. The MIPS architecture and instruction set are described in
 extenso in Appendix A of Patterson and
	 Hennessy~\cite{PattersonHennessy} (available online).}
instruction set, as the type \cd{mips}:

\medskip
%\begin{minipage}{0.46\textwidth}
%\end{code}
%\end{minipage}
%\hfill
%\begin{minipage}{0.46\textwidth}
%\begin{code}[frame=lines,label=\textit{MIPS.sml}]
\begin{code}[frame=lines,label=\textit{MIPS.sml}]
datatype mips
 = LABEL of string
 | EQU of string*string
 | GLOBL of string
 | TEXT of string
 | DATA of string
 | SPACE of string
 | ASCII of string
 | ASCIIZ of string
 | ALIGN of string
 | COMMENT of string
 | LA of string*string
 | LUI of string*string
 | ADD of string*string*string
 | ADDI of string*string*string
 | SUB of string*string*string
 | MUL of string*string*string (* low bytes of result in dest., no overflow *)
 | DIV of string*string*string (* quotient in dest., no remainder *)
 | AND of string*string*string
 | ANDI of string*string*string
 | OR of string*string*string
 | ORI of string*string*string
 | XOR of string*string*string
 | XORI of string*string*string
 | SLL of string*string*string
 | SRA of string*string*string
 | SLT of string*string*string
 | SLTI of string*string*string
 | BEQ of string*string*string
 | BNE of string*string*string
 | BGEZ of string*string
 | J of string
 | JAL of string * string list (* label + arg. reg.s *)
 | JR of string * string list (* jump reg. + result reg.s *)
 | LW of string*string*string (* lw rd,i(rs), encoded as LW (rd,rs,i) *)
 | SW of string*string*string (* sw rd,i(rs) encoded as SW (rd,rs,i) *)
 | LB of string*string*string (* lb rd,i(rs) encoded as LB (rd,rs,i) *)
 | SB of string*string*string (* sb rd,i(rs) encoded as SB (rd,rs,i) *)
 | NOP
 | SYSCALL
\end{code}

We explain in brief the meaning of each instruction.

\begin{description}


\item[\texttt{LABEL}]
defines a \emph{label} (a target for jump instructions).

\item[\texttt{EQU}] 
  serves to define constants in assembly. For instance, \cd{EQU~("x","-16")}
  stands for the assembly instruction \cd{x = -16}. 
  The two fields are strings because one can use hexadecimal format 
  and other (previously defined) constants and labels as values.

\item[\texttt{ GLOBL, TEXT, DATA, SPACE, ASCII, ASCIIZ} and \texttt{ALIGN}] 
  stand for the assembly directives 
    \cd{.globl, .text, .data, .space, .ascii, .asciiz} and \cd{.align}. 
    They delimit different segments of an assembly program, and are used by the
    assembler which creates machine code from it.

\item[\texttt{COMMENT}] contains a comment.

\item[\texttt{LA}] is the instruction to load an address into a register.
	The register is given as either a number between 0 and 31 (as a string)
	or the name of a variable that is held in a numerical register.
	\emph{Symbolic register names} such as \cd{v0} or \cd{a0} \emph{cannot be used},
	because everything that is not a number will be considered as an allocated
	variable by the register allocator later.

\item[\texttt{LUI}] loads a value (`I' for ``immediate'') into the upper
	16 bit of a register (given as a number between 0 and 31 or a name, as 
	described above).

\item[\texttt{ADD} \ldots \texttt{SLTI}]
	are common binary arithmetic, logical and shift operations
	(using registers or, with `I', values) as MIPS instructions.
	Detailed descriptions of these instructions can be looked up 
	in other sources (for instance Appendix A of Patterson and
	 Hennessy~\cite{PattersonHennessy}).
	Register arguments are given as described above, numeric constants 
	are either (signed) decimal numbers, hexadecimal numbers starting 
	with \cd{0x}, or symbolic constants defined by \texttt{EQU} before.

\item[\texttt{BEQ}\ldots\texttt{J}] 
	are conditional and unconditional jump instructions which take a label
	argument (and one or two register arguments in case of branch instructions).
    Execution continues at the instruction designated 
	by the label argument. In case of branch instructions, the register
	values are checked for the indicated condition before, and continues
	with the next instruction if the condition is false (for instance,
	\emph{branch-equal} only jumps if the values in the two argument registers
	are equal).

\item[\texttt{JAL},]
	the MIPS \emph{jump-and-link} instruction \cd{jal}, stores the address
	of the instruction that follows \cd{jal} into the return address register \cd{ra}
	and execution continues at the instruction designated by the argument label.

    In order to do register allocation on MIPS code, \cd{JAL} should, besides the jump
	target label, carry a list of variables and registers that are live at the destination
	(typically argument registers and global variables).
    The \cd{JAL} instruction in register-allocated code does not use this list and it 
    can thus be left empty.
    

\item[{\tt JR},] the MIPS \emph{jump-register} instruction {\tt jr}, continues
	execution at the address stored in the register that is given as an argument.
%
	Besides the jump target register, \cd{JR} carries a list of live variables and 
	registers at the destination (typically return value registers and global variables).
    The \cd{JR} instruction in register-allocated code does not use this list and it 
    can thus be left empty.

\item[{\tt LW, SW, LB {\rm and} SB}] are the MIPS \emph{load} and \emph{store} instructions
	for words and bytes (\emph{load/store-word/byte}, \cd{lw, sw, lb, sb}).
	Please note: in contrast to usual MIPS assembly format, the \emph{offset} from
	the source address stands \emph{last} in the parameter list. For instance, 
	\cd{LW~("2","28","16")} encodes the instruction \cd{lw~\$2,16(\$28)} (load word 
	from address in register 28 plus 16 byte offset into register 2).

\item[{\tt NOP}] is the empty instruction \cd{nop}, \emph{no operation}.

\item[{\tt SYSCALL}] is the \emph{system call} instruction \cd{syscall}.
	Register \$2 contains the system operation to execute, parameters are given
	in registers \$4 and \$5, a possible return value will appear in \$2.

	System calls interface to operating system services and will not be 
	explained here. Interested readers are referred to appendix A.9 in
	\cite{PattersonHennessy}.

\end{description}

\noindent
Aside from these instructions, \cd{Mips.sml} also provides two pseudo-instructions
which are part of MIPS, but implemented by an \emph{immediate-or} instruction:

\begin{description}

\item[{\tt MOVE (x,y)}] (MIPS \cd{move}) is translated to {\tt ORI (x,y,"0")},
	a bitwise \cd{OR} of $0$ with the source register \cd{y}, storing in \cd{x}.
	The register allocator can remove a \cd{move} instruction if target
	and source (\cd{x} and \cd{y}) are allocated to the same register.

\item[{\tt LI (x,k)}], the MIPS \emph{load-immediate} instruction \cd{li}, 
	is translated to {\tt ORI (x,"0",k)}, a bitwise \cd{OR} of $0$ with 
	the constant \cd{k}.

\end{description}

\noindent
A MIPS assembly program is represented simply as a list of these instructions, i.e. 
\cd{type MipsProgram = Mips.mips~list}. 

Finally, the {\tt Mips} module defines
functions to "pretty-print" MIPS instructions, and a function \cd{pp\_mips\_list}
that prints an entire program. The resulting string can then be written to a file 
and later read in by a MIPS assembler or simulator.

\section{The Register Allocator}
\subsection{Interface and Functionality}
Module  \cd{RegAlloc.sml} provides a register allocator, which translates 
symbolic register names into register numbers. The allocator works on 
the body of a single function at a time. Its type is

\begin{code}[frame=lines, label=from \textit{RegAlloc.sig}]
  val registerAlloc : 
    Mips.mips list -> string list -> int -> int -> int -> int
    -> Mips.mips list * string list * int * int
\end{code}

% updated to describe spilling capabilities
The arguments of \cd{registerAlloc} are a list of MIPS instructions,
a list of registers that are live \emph{after} executing the instructions,
and four register numbers: the lowest usable register (typically 2 for MIPS),
the largest \emph{caller-saves} register, the maximum register overall
(MIPS has 32 registers, but some of them serve special purposes. We use 25),
and the number of registers that have already been \emph{spilled}\footnote{
\emph{Variable spilling} means to store a variable in memory instead of holding
it in a register. It occurs when the target processor cannot provide enough
registers to allocate one for each variable used throughout the code.}
in the code sequence (for recursive calls, should initially be zero).
Registers in the range from the first to the second number are \emph{caller-saves}
registers, while those with numbers between the second and the third number are
\emph{callee-saves}. The standard conventions for MIPS are 2, 15 and 25.

{\tt registerAlloc} returns a quadrupel. The first, and most important, component 
is a modified list of instructions with symbolic register names replaced by 
numerical registers\footnote{For better readability and debugging, the instruction 
	sequence is also commented with the original symbolic instruction.}, and 
\cd{move} instructions removed whenever source and target register were allocated
to the same numerical register.

The other return values are: a list of variables that are live at entry of the code,
the maximum register number allocated in the instruction sequence, and the number of
variables that had to be spilled.
The live variables are only for debugging, the maximum register number
is needed for generating code that protects \emph{callee-saves} registers.
The number of spilled variables is used to compute how much extra stack space
must be provided for those variables when executing the resulting code sequence.
%
%The register allocator does not \emph{spill} registers, as this would make it
%considerably more complicated. If there are not enough registers available,
%an exception \texttt{not\_colourable} is raised instead.

\subsection{Using the register allocator in function translation}

We use a simplified version of the MIPS calling conventions:

\begin{itemize}

\item Registers \$2 \ldots \$15 are {\em caller-saves} and \$16 \ldots
  \$25 are {\em callee-saves}.

\item Parameters are passed in register \$2 \ldots \$15 (as necessary)
	and the function result is passed in register \$2.
	We assume that there are never more than 14 parameters, ensuring that
	all parameters can be passed in registers.

\item Register \$29 is the stack pointer. The stack grows towards zero 
	(in negative direction), and the stack pointer (\cd{SP}) points to
	the top-most element (with the lowest address).
	No \emph{frame pointer} is used.

\end{itemize}

\subsubsection{Function Calls}
The register allocator implements \emph{caller-saves} for jumps, 
and will not allocate variables to \emph{caller-saves} registers if
these are live after the function call. Thus, a function call such as
\texttt{t = f(x,y,z)} can be translated to the following instructions:

\begin{code}
[Mips.MOVE ("2","x"),
 Mips.MOVE ("3","y"),
 Mips.MOVE ("4","z"),
 Mips.JAL ("f",["2","3","4"]),
 Mips.MOVE ("t","2")]
\end{code}

Please note that the list of registers used as parameters in the call is
an extra argument to the \cd{jal} instruction (as explained above).
When more than 14 parameters have to be used, remaining parameters should 
be passed on the stack.

\subsubsection{Prolog and Epilog of Functions}

Once we have translated the \emph{body} of a function definition 
\cd{f(x,y,z) = $e$} (the part that computes the value of $e$ from 
\cd{x}, \cd{y} and \cd{z}) to a list of MIPS instructions, this code
sequence must be wrapped by additional instructions to receive the
arguments in registers, and to return the result in register \$2.
Assuming the code sequence \cd{body} for $e$ stores the value in 
a symbolic register \cd{result}, the
entire body of \cd{f} is the following:

\begin{code}
val body_symb = [Mips.MOVE ("x","2"),
                 Mips.MOVE ("y","3"),
                 Mips.MOVE ("z","4")]
                @ body @
                [Mips.MOVE ("2","result")]
\end{code}

This symbolic code is now run through the register allocator:

\begin{code}
val (f_body, _, maxReg, spilled) =
        RegAlloc.registerAlloc body_symb ["2"] 2 15 25
\end{code}
The arguments indicate that:
\begin{itemize}
\item register \$2 is the only live register at the exit of \cd{body\_symb}.
	It will contain the return value.
\item The minimum register is 2, the highest caller-saves register is 15,
	 and the maximum register available is 25.
\end{itemize}

The returned \cd{f\_body} contains the body of funtion \cd{f} with numerical
registers. Many of the \cd{move} instructions we added in the previous step
are likely to be removed by the register allocator, as it will try to use 
the registers which are already filled with the right values.

Return value \cd{maxReg} indicates the highest register number used in 
\cd{f\_body}. If this number is higher than 15 (maximum caller-saves),
the body uses \emph{callee-saves registers} which need to be 
\emph{saved to the stack} before executing the body code and 
\emph{restored afterwards}.
If the function body requires more registers than available, variables are
spilled to the top addresses of the stack. To reserve stack space for this
spilling, \cd{spilled}$\cdot$4 bytes at the top are kept free,
in addition to space for the callee-saves registers below the spilling area.
We also save the return address from register \cd{RA} (register \$31) to the 
stack, as the body might call other functions and overwrite the address.

For instance, if \cd{maxReg} is 17 and \cd{spilled} is 2, the stack should grow by
8 bytes for spilling, 8 bytes for registers \$16 and \$17 as well as 4 bytes for 
\cd{RA} (\$31), thus 20 bytes in total. 
While spilling happens inside the function body, callee-saves registers are saved 
to the stack upon entry and restored at the end.
In summary, this leads to the following code:

\medskip
\begin{minipage}{0.6\textwidth}
\begin{code}
val prologue = [Mips.LABEL "f",
                Mips.SW ("31","29","-4"),   # return address (RA)
                Mips.ADDI ("29","29","-20") # adjust SP
                Mips.SW ("17","29","12"),   # 17, callee-saves 2
                Mips.SW ("16","29","8"),    # 16, callee-saves 1

val epilogue = [Mips.LW ("17","29","12"),   # callee-saves 2
                Mips.LW ("16","29","8"),    # callee-saves 1
                Mips.ADDI ("29","29","20"), # adjust SP
                Mips.LW ("31","29","-4"),   # return address (RA)
                Mips.JR ("31",[])]

val f_all = prologue @ f_body @ epilogue
\end{code}
\end{minipage}
\hfill
\begin{minipage}{0.18\textwidth}
{\footnotesize
\setlength{\unitlength}{0.27em}
\begin{picture}(25,62)
%\put(0,0){\dashbox(25,55){}}

\thicklines
\put(0,25){\vector(0,1){20}}
\put(0,28){\rotatebox{90}{growing}}

\put(5,52){\dashbox(20,12)[t]{\it Free}}
\put(10,58){\it Stack}
\put(7,54){\it (garbage)}

\put(5,40){\framebox(20,12){\it Spilling}}
\put(12,42){\it area}

\put(5,34){\framebox(20,6){\tt saved \$16}}
\put(5,28){\framebox(20,6){\tt saved \$17}}
\put(5,22){\framebox(20,6){\tt saved RA}}

\put(5,10){\framebox(20,12)[t]{\it previous}}
\put(10,15.5){\it call's}
\put(10,11){\it stack}
\put(5,3){\framebox(20,7){..}}

\put(-2,52){\vector(2,0){6}}
\put(-2,53){\tt SP}

\thinlines
\put(0,22){\line(1,0){30}}
\put(0,10){\line(1,0){30}}

\end{picture}
}
\end{minipage}

The stack pointer is (by convention) always stored in register \$29,
and the stack grows towards smaller addresses. The picture on the right shows the
stack immediately after entering the function body. The body instructions might
call other functions (or \cd{f}, recursively) and make the stack grow 
further.

The return address register (sometimes called \emph{link register}, used
by instruction \cd{JAL}) is \$31.
Note that this is register-allocated code, so the list of live registers 
for the \cd{JR} instruction can be empty.

The previous example is for a specific function with 3 arguments that uses
two callee-saves registers ({\tt maxReg}) and reserves space for two spilled
variables. The code generator of a compiler needs 
to generate prolog and epilog code for a varying number of arguments,
callee-saves registers, and spilling.
%
%Figure~\ref{translateFunction} shows a sketch of how such code might look
%like. If a function uses more parameters than available registers,
%the helper function {\tt movePars} must be extended to pass some arguments
%on the stack instead.
%
%\begin{figure}[tbh]
%\begin{code}[frame=lines,label=\textrm{translating a function}]
%val SP = "29"
%val RA = "31"
%
%fun translateFunction(name, parameters, bodyExp, ftable) =
%  let val body = translateExp(bodyExp, emptyTable, ftable, "2")
%      val body_symb = movePars parameters 2 @ body
%      val (f_body, _, maxReg) =
%            RegAlloc.registerAlloc body_symb ["2"] 2 15 25
%      val (saveCode, restoreCode, frameSize) = saveRestore maxReg
%      val prologue = [Mips.LABEL name, Mips.SW (RA,SP,"-4")]
%                     @ saveCode @
%                     [Mips.ADDI (SP,SP,Int.toString (~frameSize))]
%      val epilogue = [Mips.ADDI (SP,SP,Int.toString frameSize)]
%                     @ restorecode @
%                     [Mips.LW (RA,SP,"-4"), Mips.JR (RA,[])]
%  in prologue @ f_body @ epilogue
%  end
%
%and movePars [] _ = []
%  | movePars (par::pars) reg =
%      Mips.MOVE (par,Int.toString reg) :: movePars pars (reg+1)
%
%and saveRestore reg =
%  if reg<16 then ([],[],4)
%  else let val (save,restore,size) = saveRestore (reg-1)
%           val newSize = size+4
%           val args = (Int.toString reg, SP, int.toString (~newSize))
%       in (Mips.SW args :: save, Mips.LW args :: restore, newSize)
%       end
%\end{code}
%\vspace*{-2em}
%\caption{Code for translating functions}
%\label{translateFunction}
%%\hrulefill
%\end{figure}

\section{MARS, a MIPS simulator}

MARS is an acronym for \emph{MIPS Assembler and Runtime Simulator}, and supports
a large subset of MIPS instructions. MARS is compatible with the 
SPIM simulator described in Appendix A.9 of Patterson/Hennessy~\cite{PattersonHennessy}.
MARS is written in Java and available under the URL
\url{http://courses.missouristate.edu/kenvollmar/mars/}.
Its current version is 4.4 (released in August 2013).
An introductory manual can be found at 
\url{http://courses.missouristate.edu/KenVollmar/MARS/Help/MarsHelpIntro.html}.

MARS can be used to directly assemble and execute a MIPS assembly program, 
by the following command on the command line:
\begin{code}[fontsize=\normalsize]
me@home$ java -jar Mars4_4.jar program.asm
\end{code}
%$
The executed program will use standard input and output to interact with the user.

MARS also comes with a rich GUI where the machine state during program execution can 
be observed and the assembly code edited. The GUI starts if MARS is started without
an additional argument, like so:
\begin{code}[fontsize=\normalsize]
<<<<<<< HEAD
me@home$ java -jar Mars_4_2.jar
=======
me@home$ java -jar Mars4_4.jar
>>>>>>> 36fa5463c25d7becd15b1dedc278d08e69956910
\end{code}
%$

The following things should be noted when using MARS:

\begin{itemize}
\item Registers \$1, \$26 and \$27 are reserved for expanding pseudo-instructions
	(such as \cd{la}) and for the operating system, respectively.

\item The stack pointer (\$29) is initialised at program start and does not 
	need re-initialisation.

\item The stack pointer points to the top element, so new stack elements need to be
	written to the address \emph{4 bytes below} the current SP (offset 4 on 
	32-bit architectures).

\item Global variables, tables, and other data created at runtime, should be
	placed in a data segment of the assembly, and grow upward in memory from the
	start of the data segment.

\end{itemize}

\begin{thebibliography}{1}

\bibitem{PattersonHennessy}
David~A. Patterson and John~L. Hennessy.
\newblock {\em Computer Organization \& Design, the Hardware/Software
  Interface}.
\newblock Morgan Kaufmann, 1998.
\newline
\newblock Appendix A is freely available at 
\url{http://www.cs.wisc.edu/~larus/HP_AppA.pdf}.

\end{thebibliography}

\end{document}
