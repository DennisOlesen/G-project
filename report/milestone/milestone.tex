\documentclass[11pt,a4paper]{article}

\usepackage[utf8]{inputenc}
\usepackage[danish]{babel}
\usepackage[T1]{fontenc}
\usepackage{pdfpages}

\usepackage{amsmath,amssymb,amsfonts}

\let\oldemptyset\emptyset
\let\emptyset\varnothing

\begin{document}
\section*{Parser implementering}
\label{sec:implementation}
\subsection*{Tokens (terminaler)}
Tokens svarer til terminaler i mosmlyac og for at kende bedre forskel
på vores terminaler og non-terminaler, indførte vi en konvention om,
at alle tokens skulle skrives i uppercase. Konventionen betød, at
actions i Lexer.lex skulle opdateres, så LL1Parser.Token blev til
Parser.TOKEN. Derudover skrev vi token typer som ML standard
typer og ikke som beskrevet i AbSyn, da AbSyn åbnes i headeren,
som først starter efter token definitionerne.
\subsection*{Typer (non-terminaler)}
Typer forstås som non-terminaler, og de har fået giver typer, der
svarer til deres beskrivelse i AbSyn.sml. For nogle af typerne,
var der ikke direkte defineret en tilsvarende type i AbSyn.sml.
I sådanne tilfælde udledte vi typen ud fra grammatikken. Fx gav
vi FunDecs typen FunDec list.
\subsection*{Associativitet og forrang}
I mosmlyac har associativiteterne større associativitet jo længere
nede i filen de står. Dette udnyttede vi til fx at give gange og divider
en højere associativitet end plus og minus. Det samme gøre sig gælde for
de boolske operatorer and, or og not. Operatorerne plus, minus, gange,
dividere, or og and blev desuden gjort venstre associative jævnfør
almindelige matematiske regler. Not operatoren blev gjort højre
associativ. De komparative operatorer lig med og mindre end blev gjort
ikke-associative. Som en konsekvens heraf, tillader vi ikke udtryk på
formen $x_{1} < \dots < x_{n}$ eller $x_{i} = \dots = x_{n}$, hvor
$n > 2$. Jævnfør bogen side 51 løste vi ``the dangling else problem''
ved at gøre terminalerne IF, THEN og ELSE højre associative.
\subsection*{Regler (produktioner)}
I reglerne måtte vi lave enkelte omskrivninger fra den struktur,
som er givet i opgave beskrivelsen. Herunder måtte vi slette Op produktionerne
og skrev i stedet for højresiderne ind i Exp produktionerne.
Således  blev eksempelvis
\begin{align*}
    Op \rightarrow Exp + Exp
\end{align*}
til
\begin{align*}
    Exp \rightarrow Exp + Exp
\end{align*}
Dette gjorde vi for operatorerne: plus, minus, gange, dividere
and, or, lig med og mindre end. Endeligt lavede vi også om
i blok strukturen. I den udleverede grammatik findes der produktioner
for de tre nonterminaler: Block, DBlock og SBlock. Problemet med den
opdeling af blok strukturen er, at nonterminalerne i DBlock ikke kunne
vide, hvornår SBlock begyndte. Det var derfor ikke muligt at have nogle
variabel erklaringer i en procedure eller function. Løsning blev at
fjerne opdelingen i blokke og kun beholde Block, som da blev repræsenteret
ved følgende grammatik.
\begin{align*}
    Block &\rightarrow \text{VAR} \; Decs \; \text{; BEGIN} \; StmtSeq \; \text{; END}\\
    &\text{|} \; \text{VAR} \; Decs \; \text{;} \; Stmt\\ 
    &\text{|} \; \text{BEGIN} \; StmtSeq \; \text{; END} \; Stmt\\ 
    &\text{|} \; Stmt
\end{align*}
\section*{Test}
Da vi endnu ikke har implementeret type tjek, herunder de boolske operatorer
or og not, så er det begrænset, hvor mange af de leverede filer i DATA mappen,
vi kan bruge til at teste. Derfor har vi valgt selv at skrive nogle testtilfælde,
hvor vi tester for: addition/subtraktion, boolsk and operator, funktioner,
parameter lister og blok struktur. De filer vi har lavet er: arithmetic.pal,
boolean.pal, declarations.pal, functions.pal. Når vi tester for blok struktur,
skal det forstås sådan, at vi, som beskrevet i forrige afsnit, oplevede
udfordringer i forhold DBlocks og SBlocks. Vi ønskede derfor at være sikre på, at 
løsningen virkede optimalt. Alle vores egne filer blev oversat uden fejl, mens
det kun er følgende af de leverede filer, der blev oversat uden fejlmeddeleser:
fibRec.pal, fibWhile.pal, proctest.pal, readtest.pal og shortest.pal. De filer,
der ikke blev oversat, skyldes, at vi endnu ikke har løst opgave 2, 3 og 4.
\section*{Ideer og status}
\subsection*{1. Parser}
Denne er færdig og står beskrevet i detaljer i afsnittet "Parser implementering"
\subsection*{2. Operatorer}
Gange, divider, or og not er alle sammen indsat i Parser.grm og Lexer.lex.
Dette er de 4 operationer, som er i fokus i de resterende delopgaver,
og vi forventer de først at kan testes, når alle opgaverne er færdige.
Vi forventer deraf også, at skulle arbejde på denne delopgave sideløbende
med de andre delopgaver.
\subsection*{3. Type inference}
Den er ikke påbegyndt, men det ligger i forlængelse af ugens pensum, og vi kan se af type.sml,
at typecheckeren ikke er lavet, så dette skal laves i stil med de andre operationer,
som i forvejen findes i filen. Vi regner med at lave type inference ud fra
lignende teknikker, som der blev præsenteret til forelæsningen onsdag d. 4/12.
Dvs., at hvis vi fx har udtrykket ``a = read() + read()'', så skal ``a'' have typen
int.
\subsection*{4. Array indeksering}
Denne delopgave er endnu ikke påbegyndt, men vi forventer
at kunne løse type check delen baseret på erfaringer og
teknikker fra delopgave 2 og 3.
\subsection*{5. Call-by-value}
Denne opgave er endnu ikke påbegyndt.
\end{document}

