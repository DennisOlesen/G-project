\documentclass[11pt,a4paper]{article}

\usepackage[utf8]{inputenc}
\usepackage[danish]{babel}
\usepackage[T1]{fontenc}
\usepackage{pdfpages}

\usepackage{amsmath,amssymb,amsfonts}

\let\oldemptyset\emptyset
\let\emptyset\varnothing

\begin{document}
\section*{Parser implementering}
\label{sec:implementation}
Tokens:\\
Tokens er tilsvarende terminaler i mosmlyac, og for at kende bedre forskel på vores terminaler og non-terminaler indførte vi en konvention om at alle tokens blev skrevet i caps.\\
Derudover skrev vi token typer som ML standard typer og ikke som beskrevet i AbSyn, da AbSyn åbnes i headeren, som først starter efter token definitionerne.\\
Tokens er lavet så de svarer til de tokens lexeren kan danne. Dette betød samtidig at vi valgte at omskrive lexerens tokens fra Ttoken til TOKEN.\\
\\
Typer:\\
Typer forstås som non-terminaler, og de har fået typer tilsvarende deres beskrivelse i AbSyn, og dem der ikke kunne findes der, havde noget i deres regler, som vi derfra kunne komme frem til hvad typen skal være.
\\
Associativitet:\\
I mosmlyac har associativiteterne større associativitet jo længere nede i filen de står. Dette udnyttede vi til at give, f.eks. gange og divider, en højere associativitet end f.eks. plus og minus. Derudover tilføjede vi også associtiativitet til de terminaler der ellers skulle bruge det. Herunder If then og else som skulle gøres højre assoc.\\
Regler:\\
I reglerne måtte vi lave enkelte omskrivninger fra den struktur som er givet i opgave beskrivelsen. Herunder er bl.a. \\
Op: Exp Op Exp\\
Som rykkede over i Exp, og lavede:\\
Exp: Exp PLUS Exp\\
   | Exp MINUS Exp\\
... og ligeledes for de resterende operatore.
Integrering:\\
For at integrere den i compileren ændrede vi alle tilfælde af LL1Parser.Ttoken... til Parser.TOKEN i lexeren.



\section*{Test}
Da vi endnu ikke har implementeret type tjek, herunder de boolske operatorer
or og not, så er det begrænset, hvor mange af de leverede filer i DATA mappen,
vi kan bruge til at teste. Derfor har vi valgt selv at skrive nogle testtilfælde,
hvor vi tester for: addition/subtraktion, boolsk and operator, funktioner,
parameter lister og blok struktur. De filer vi har lavet er: arithmetic.pal,
boolean.pal, declarations.pal, functions.pal. Når vi tester for blok struktur,
skal det forstås sådan, at vi, som beskrevet i (\ref{sec:implementation}), oplevede
udfordringer i forhold DBlocks og SBlocks. Vi ønskede derfor at være sikre på, at 
løsningen virkede optimalt. Alle vores egne filer blev oversat uden fejl, mens
det kun er følgende af de leverede filer, der blev oversat: fibRec.pal,
fibWhile.pal, proctest.pal, readtest.pal og shortest.pal. De filer, der ikke
blev oversat, skyldes, at vi endnu ikke har løst opgave 2, 3 og 4.

\section*{Ideer og Status}

\subsection{1}
Denne er færdig og står beskrevet i detaljer i afsnittet "Parser implementering"

\subsection{2}
Gange, divider, or og not er alle sammen indsat i parser og lexer. Dette er de 4 operationer som er i fokus i de resterende delopgaver, og de forventes først at kunne compiles når alle opgaverne er færdige.
Dette betyder at vi imiderletid fortsætter til delopgave 3.

\subsection{3}
Den er ikke påbegyndt men det ligger i forlængelse af ugens pensum, og vi kan se af type.sml at typecheckeren ikke er lavet så dette skal laves i stil med de andre operationer som i forvejen er i filen.
=======
\section*{Ideer}

\subsection{4}
nope
\subsection{5}
not even a little bit.
\end{document}

