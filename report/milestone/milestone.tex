\documentclass[11pt,a4paper]{article}

\usepackage[utf8]{inputenc}
\usepackage[danish]{babel}
\usepackage[T1]{fontenc}
\usepackage{pdfpages}

\usepackage{amsmath,amssymb,amsfonts}

\let\oldemptyset\emptyset
\let\emptyset\varnothing

\begin{document}
\section*{Parser implementering}
Tokens:\\
Tokens er tilsvarende terminaler i mosmlyac, og for at kende bedre forskel på vores terminaler og non-terminaler indførte vi en konvention om at alle tokens blev skrevet i caps.\\
Derudover skrev vi token typer som ML standard typer og ikke som beskrevet i AbSyn, da AbSyn åbnes i headeren, som først starter efter token definitionerne.\\
Tokens er lavet så de svarer til de tokens lexeren kan danne. Dette betød samtidig at vi valgte at omskrive lexerens tokens fra Ttoken til TOKEN.\\
\\
Typer:\\
Typer forstås som non-terminaler, og de har fået typer tilsvarende deres beskrivelse i AbSyn, og dem der ikke kunne findes der, havde noget i deres regler, som vi derfra kunne komme frem til hvad typen skal være.
\\
Associativitet:\\
I mosmlyac har associativiteterne større associativitet jo længere nede i filen de står. Dette udnyttede vi til at give, f.eks. gange og divider, en højere associativitet end f.eks. plus og minus. Derudover tilføjede vi også associtiativitet til de terminaler der ellers skulle bruge det. Herunder If then og else som skulle gøres højre assoc.\\
Regler:\\
I reglerne måtte vi lave enkelte omskrivninger fra den struktur som er givet i opgave beskrivelsen. Herunder er bl.a. \\
Op: Exp Op Exp\\
Som rykkede over i Exp, og lavede:\\
Exp: Exp PLUS Exp\\
   | Exp MINUS Exp\\
... og ligeledes for de resterende operatore.

Integrering:\\
For at integrere den i compileren ændrede vi alle tilfælde af LL1Parser.Ttoken... til Parser.TOKEN i lexeren.



\section*{Test}

\section*{Ideer}

\end{document}

